\documentclass[twoside,11pt]{article}

% Any additional packages needed should be included after jmlr2e.
% Note that jmlr2e.sty includes epsfig, amssymb, natbib and graphicx,
% and defines many common macros, such as 'proof' and 'example'.
%
% It also sets the bibliographystyle to plainnat; for more information on
% natbib citation styles, see the natbib documentation, a copy of which
% is archived at http://www.jmlr.org/format/natbib.pdf

\usepackage{jmlrwcp2e}
\usepackage{url}

% Definitions of handy macros can go here

\newcommand{\dataset}{{\cal D}}
\newcommand{\fracpartial}[2]{\frac{\partial #1}{\partial  #2}}

% Heading arguments are {volume}{year}{pages}{submitted}{published}{author-full-names}

\jmlrheading{1}{2007}{1-16}{}{}{Author Name1 and Author Name2}{NIPS 2008 workshop on causality}

% Short headings should be running head and authors last names

\ShortHeadings{Short Title}{Author Surnames}
\firstpageno{1}

\begin{document}

\title{Full Paper Title}

\author{\name Author Name1 \email author1@institute.edu \\
       \addr First Line\\
       Second Line\\
       City, State, Zip Code, Country
       \AND
       \name Author Name2 \email author2@institute.edu \\
       \addr First Line\\
       Second Line\\
       City, State, Zip Code, Country}

\editor{Isabelle Guyon, Dominik Janzing and Bernhard Sch\"olkopf}

\maketitle

\begin{abstract}%   <- trailing '%' for backward compatibility of .sty file
This is the abstract.
\end{abstract}

\begin{keywords}
  Your keywords here, comma separated
\end{keywords}

\section{Introduction}
\label{sec:introduction}

Please limit your paper to {\bf 6 pages} unless you have been instructed otherwise. Submit by November 21, 2008 to causality@clopinet.com. The papers will be pier reviewed and at revision time the number of pages might be increased. The challenge participants should {\bf append their fact sheet} and may {\bf add one extra page} for that purpose. \\

{\noindent \em Remainder omitted in this sample. See http://www.jmlr.org/papers/ for full paper.}

% Acknowledgements should go at the end, before appendices and references

\acks{We would like to acknowledge support for this project ...}

% Manual newpage inserted to improve layout of sample file - not
% needed in general before appendices/bibliography.
\bibliography{my_bib}

\newpage

\appendix
\section*{Appendix A. A theorem}
\label{sec:theorem}

% Note: in this sample, the section number is hard-coded in. Following
% proper LaTeX conventions, it should properly be coded as a reference:

%In this appendix we prove the following theorem from
%Section~\ref{sec:textree-generalization}:

In this appendix we prove the following theorem from
Section~1.1:

\noindent
{\bf Theorem} {\it Let ...} \hfill\BlackBox

\noindent
{\bf Proof}. We use the notation:

\section*{Appendix B. Pot-luck challenge: FACT SHEET.}
\label{sec:factsheet_v1}

{\begin{center}  \em (for a task solved) \\ \end{center} }

\noindent {\bf Title:}\\
\noindent {\bf Participant name, address, email and website:}\\
\noindent {\bf Task(s) solved:}\\
\noindent {\bf Reference:} Provide a pointer to a technical memorandum or a paper (optional).\\

\noindent {\bf Method:}\\
Summarize the algorithms you used in a way that those skilled in the art should understand what to do. Profile of your methods as follows:
\begin{itemize}
\item Preprocessing
\item Causal discovery
\item Feature selection
\item Classification
\item Model selection/hyperparameter selection\\
\end{itemize}

\noindent {\bf Results:} The reader should also know from reading the fact sheet what the strength of the method is. To that end, provide a result table:

\begin{table}[h]
\begin{center}
\label{tab:table1}
\begin{tabular}{|c|c|c|}
\hline
Dataset/Task & Score 1 & Score 2 \\
\hline
CYTO &  & \\
PROMO & & \\
etc. & & \\
\hline

\end{tabular}
\caption{Result table.}
\end{center}
\end{table}

Comment about the following:
\begin{itemize}
\item quantitative advantages (e.g. compact feature subset, simplicity, computational advantages)
\item qualitative advantages (e.g. compute posterior probabilities, theoretically motivated, has some elements of novelty).
\end{itemize}
Briefly explain your implementation. Provide a URL for the code (if available). Precise whether it is a push-button application that can be run on benchmark data to reproduce the results, or resources such as modules or libraries. \\

\noindent {\bf Keywords:}
Put at least one keyword in each category. Try some of the following keywords and add your own:
\begin{itemize}
\item Preprocessing or feature construction: centering, scaling, standardization, PCA.
\item Causal discovery: Bayesian Network, Structural Equation Models, Probabilistic Graphical Models, Markov Decision Processes, Propensity Scoring, Information Theoretic Method.
\item Feature selection: filter, wrapper, embedded feature selection, feature ranking, etc.
\item Classifier: neural networks, nearest neighbors, tree classifier, RF, SVM, kernel-method, least-square, ridge regression, L1 norm regularization, L2 norm regularization, logistic regression, ensemble method, bagging, boosting, Bayesian, transduction.
\item Hyper-parameter selection: grid-search, pattern search, evidence, bound optimization, cross-validation, K-fold.
\item Other: ensemble method, transduction.
\end{itemize}

\section*{Appendix C. Pot-luck challenge: FACT SHEET .}
\label{sec:factsheet_v2}

{\begin{center} \em (for a donated dataset)\\ \end{center}}

\noindent {\bf Repository URL:}  \url{http://www.causality.inf.ethz.ch/repository.php?id=<yournum>}\\

{\noindent \em Include below the information shown at the above URL.}\\

\noindent {\bf Dataset name:} \\

\noindent {\bf Title:} \\
\noindent {\bf Authors:} \\
\noindent {\bf Contact name, address, email and website:} \\

\noindent {\bf Key facts:} \\
Data dimensions (number of variables, number of entries), variable types, missing data, etc. See \url{ftp://ftp.ics.uci.edu/pub/machine-learning-databases/DOC-REQUIREMENTS} for inspiration.\\

\noindent {\bf Abstract:} \\

\noindent {\bf Keywords:} \\

\vskip 0.2in
\bibliography{sample}

\end{document}
